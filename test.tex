\documentclass[11pt, draft]{article}

\usepackage[utf8]{inputenc}
\usepackage[margin=35truemm]{geometry} % 余白を35mmに設定

% --- 数学関連のパッケージ ---
\usepackage{amsmath}      % 様々な数式環境
\usepackage{amsthm}       % 定理環境
\usepackage{amsfonts}     % 数学用フォント
\usepackage{physics}      % 物理学の数式用マクロ (bra, ket, dvなど)
\usepackage{bm}           % 太字の数式 \bm{}

% --- 図や色、レイアウト関連のパッケージ ---
\usepackage[dvipdfmx]{graphicx, color} % 図の挿入や色の利用
\usepackage{tikz}         % 図形描画
\usetikzlibrary{intersections,calc,arrows.meta}
\usepackage{float}        % 図表の位置を調整 [H]など
\usepackage{siunitx}      % 単位をきれいに表示 \SI{100}{\kilo\gram}など
\usepackage{ascmac}       % itemboxなどの囲み枠

% --- その他 ---
\parindent = 0pt % 全体の段落開始時のインデントをなくす

% --- ハイパーリンク関連 (できるだけ最後に読み込む) ---
\usepackage[
    dvipdfmx,
    bookmarks=true,
    bookmarksnumbered=true,
    bookmarkstype=toc
]{hyperref}
\usepackage{pxjahyper} % hyperrefの日本語文字化け対策
\begin{document}

\title{テスト文書}
\author{一ノ瀬}
\date{\today}
\maketitle


ステップ関数の微分がデルタ関数になるのは、超関数(distribution)という枠組みで微分を考えるからです。
結論から言うと、\textbf{デルタ関数を積分するとステップ関数になる}という関係性に基づいています。

\hrule
\vspace{1em}

\section{通常の微分ではなぜ不可能なのか?}

まず、ヘビサイドのステップ関数 $\Theta(x)$ は以下のように定義されます。
\[
\Theta(x) = 
\begin{cases}
1 & (x > 0) \\
0 & (x < 0)
\end{cases}
\]
この関数は $x=0$ の点において不連続であるため、通常の関数の意味では微分不可能です。
$x \neq 0$ の領域では傾きは明らかに $0$ ですが、$x=0$ での「無限大の傾き」とも言える振る舞いを数学的に厳密に記述するために、ディラックのデルタ関数 $\delta(x)$ が導入されます。

\section{超関数としての微分}

ディラックのデルタ関数 $\delta(x)$ は、以下の2つの重要な性質を持つ超関数として定義されます。

\begin{enumerate}
    \item $x \neq 0$ の領域では、常に $\delta(x) = 0$ である。
    \item 全空間で積分すると $1$ になる。
    \[
    \int_{-\infty}^{\infty} \delta(x) \,dx = 1
    \]
\end{enumerate}

この性質を持つデルタ関数を $-\infty$ からある点 $x$ まで積分すると、どうなるかを考えます。

\begin{itemize}
    \item \textbf{$x < 0$ の場合}:\\
    積分範囲にデルタ関数が値を持つ $t=0$ の点が含まれないため、積分値は $0$ です。
    \[
    \int_{-\infty}^{x} \delta(t) \,dt = 0
    \]

    \item \textbf{$x > 0$ の場合}:\\
    積分範囲に $t=0$ の点が含まれるため、デルタ関数の全ての「強さ」を拾うことになり、積分値は $1$ となります。
    \[
    \int_{-\infty}^{x} \delta(t) \,dt = \int_{-\infty}^{\infty} \delta(t) \,dt = 1
    \]
\end{itemize}

この積分結果をまとめると、まさしくステップ関数 $\Theta(x)$ の定義そのものになっていることが分かります。
\[
\int_{-\infty}^{x} \delta(t) \,dt = \Theta(x)
\]

\section{結論:微積分の基本定理}

微分と積分が互いに逆の演算であるという微積分の基本定理の関係は、超関数の世界でも成立します。
上記の積分関係式の両辺を $x$ で微分すると、
\[
\frac{d}{dx} \left( \int_{-\infty}^{x} \delta(t) \,dt \right) = \delta(x)
\]
となります。左辺の被微分関数は $\Theta(x)$ に他ならないため、以下の関係式が導かれます。
\[
\frac{d\Theta(x)}{dx} = \delta(x)
\]

これが、ステップ関数を(超関数の意味で)微分するとデルタ関数になる理由です。
\\
\\
まず、場の理論ではなく通常の解析力学(点粒子系)において、対称性のパラメータが複数個 $\epsilon^i$ ある場合へ、保存量 (Noether charge) の定義を拡張します。

\paragraph{設定}
\begin{itemize}
    \item 座標: $q^a(t)$
    \item 対称性変換: $\delta q^a = \epsilon^i h^a_i(q)$ \\
    (ここで $h^a_i(q)$ は、変換の形を決める関数です)
    \item Lagrangian $L(q^a, \dot{q}^a)$ はこの変換に対して不変 ($\delta L = 0$) であるとします(全微分項は無視)。
\end{itemize}

\paragraph{保存量 (Charge) の定義}
単一パラメータの場合の保存量 $Q = p_a \frac{\delta q^a}{\epsilon}$ を、パラメータ $\epsilon^i$ ごとの成分 $Q_i$ に一般化します。
運動量を $p_a = \frac{\partial L}{\partial \dot{q}^a}$ とすると、各 $i$ に対する保存量は以下のように定義できます。
\begin{equation}
    Q_i = p_a \frac{\partial (\delta q^a)}{\partial \epsilon^i} = \frac{\partial L}{\partial \dot{q}^a} h^a_i(q)
\end{equation}

\paragraph{保存則の確認}
これらが保存量であることを確認するために、時間微分 $\frac{d Q_i}{dt}$ を計算します。
\begin{equation}
    \frac{d Q_i}{dt} = \frac{d}{dt} \left( \frac{\partial L}{\partial \dot{q}^a} \right) h^a_i(q) + \frac{\partial L}{\partial \dot{q}^a} \frac{d}{dt} (h^a_i(q))
\end{equation}
Euler-Lagrange方程式 $\frac{d}{dt} (\frac{\partial L}{\partial \dot{q}^a}) = \frac{\partial L}{\partial q^a}$ を用いると、以下のようになります。
\begin{equation}
    \frac{d Q_i}{dt} = \frac{\partial L}{\partial q^a} h^a_i + \frac{\partial L}{\partial \dot{q}^a} \dot{h}^a_i
\end{equation}
ここで、$\delta q^a = \epsilon^i h^a_i$ および $\delta \dot{q}^a = \epsilon^i \dot{h}^a_i$ であることを思い出すと、右辺は Lagrangian の変分 $\delta L$ の係数部分になっています。
\begin{equation}
    \delta L = \frac{\partial L}{\partial q^a} \delta q^a + \frac{\partial L}{\partial \dot{q}^a} \delta \dot{q}^a = \epsilon^i \left( \frac{\partial L}{\partial q^a} h^a_i + \frac{\partial L}{\partial \dot{q}^a} \dot{h}^a_i \right)
\end{equation}
対称性があるため $\delta L = 0$ です。任意の $\epsilon^i$ に対してこれが成り立つので、括弧の中身は0でなければなりません。したがって、
\begin{equation}
    \frac{d Q_i}{dt} = 0
\end{equation}
となり、定義した $Q_i$ が保存量であることが示されました。

\subsection*{(b) 空間次元がない「世界」への適用}

次に、問題文にある場の理論のセットアップを、「空間次元がない (no spatial dimensions)」場合に適用して解釈します。

\paragraph{1. インデックス $\alpha$ の取りうる値}
通常、場の理論では $\xi^\alpha$ は時空の座標 ($\xi^0$ が時間、$\xi^1 \dots \xi^k$ が空間) を表します。
空間次元がない場合、座標は「時間」にあたる $\xi^0$ しか存在しません。
したがって、インデックス $\alpha$ が取りうる値は \textbf{$0$ のみ} です。

\paragraph{2. 各式の解釈と(a)との比較}
この極限において、場の理論の変数は点粒子の解析力学の変数に以下のように対応します。
\begin{itemize}
    \item 座標: $\xi^0 \rightarrow t$ (時間)
    \item 場: $\phi^a(\xi) \rightarrow q^a(t)$ (一般化座標)
    \item 微係数: $\partial_0 \phi^a \rightarrow \dot{q}^a$ (速度)
    \item Lagrangian密度: $\mathcal{L} \rightarrow L$ (Lagrangian)\\
    (※空間積分がないため、密度 $\mathcal{L}$ がそのまま $L$ になります)
\end{itemize}

与えられた式 (8.22), (8.23), (8.25) をこの対応で書き換えます。

\subparagraph{式 (8.22) の解釈 (Current)}
$\alpha = 0$ の場合のみを考えます。
\begin{equation}
    \epsilon^i j_i^0 \equiv \frac{\partial \mathcal{L}}{\partial (\partial_0 \phi^a)} \delta \phi^a
\end{equation}
これを点粒子の記法に直すと、
\begin{equation}
    \epsilon^i j_i^0 = \frac{\partial L}{\partial \dot{q}^a} \delta q^a
\end{equation}
したがって、カレントの第0成分 $j_i^0$ は、(a)で定義した保存量 $Q_i$ そのものであることがわかります。
\begin{equation}
    j_i^0 = \frac{\partial L}{\partial \dot{q}^a} \frac{\delta q^a}{\epsilon^i} = Q_i
\end{equation}

\subparagraph{式 (8.23) の解釈 (Conservation law)}
$\alpha$ の和は0のみなので、
\begin{equation}
    \partial_0 j_i^0 = 0
\end{equation}
これは、時間の偏微分(ここでは全微分と同じ)が0ということなので、
\begin{equation}
    \frac{d Q_i}{dt} = 0
\end{equation}
となり、(a)で確認した「電荷 (Charge) の保存」と完全に一致します。

\subparagraph{式 (8.25) の解釈 (Charge definition)}
\begin{equation}
    Q_i = \int d\xi^1 \dots d\xi^k j_i^0
\end{equation}
空間次元がない ($k=0$) ため、積分する空間座標が存在しません(積分測度が1、あるいは積分記号が不要)。よって、
\begin{equation}
    Q_i = j_i^0
\end{equation}
となります。

\paragraph{結論}
空間次元がない場合、場の理論における「Noetherカレント $j_i^\alpha$」は、解析力学における「Noetherチャージ $Q_i$」そのもの(のカレント第0成分)に帰着します。また、カレントの保存則 $\partial_\alpha j^\alpha = 0$ は、そのままチャージの保存則 $\dot{Q} = 0$ に対応します。

\end{document}
\end{document}