\documentclass[11pt, draft]{article}

\usepackage[utf8]{inputenc}
\usepackage[margin=35truemm]{geometry} % 余白を35mmに設定

% --- 数学関連のパッケージ ---
\usepackage{amsmath}      % 様々な数式環境
\usepackage{amsthm}       % 定理環境
\usepackage{amsfonts}     % 数学用フォント
\usepackage{physics}      % 物理学の数式用マクロ (bra, ket, dvなど)
\usepackage{bm}           % 太字の数式 \bm{}

% --- 図や色、レイアウト関連のパッケージ ---
\usepackage[dvipdfmx]{graphicx, color} % 図の挿入や色の利用
\usepackage{tikz}         % 図形描画
\usetikzlibrary{intersections,calc,arrows.meta}
\usepackage{float}        % 図表の位置を調整 [H]など
\usepackage{siunitx}      % 単位をきれいに表示 \SI{100}{\kilo\gram}など
\usepackage{ascmac}       % itemboxなどの囲み枠

% --- その他 ---
\parindent = 0pt % 全体の段落開始時のインデントをなくす

% --- ハイパーリンク関連 (できるだけ最後に読み込む) ---
\usepackage[
    dvipdfmx,
    bookmarks=true,
    bookmarksnumbered=true,
    bookmarkstype=toc
]{hyperref}
\usepackage{pxjahyper} % hyperrefの日本語文字化け対策
\begin{document}

\title{テスト文書}
\author{一ノ瀬}
\date{\today}
\maketitle


ステップ関数の微分がデルタ関数になるのは、超関数(distribution)という枠組みで微分を考えるからです。
結論から言うと、\textbf{デルタ関数を積分するとステップ関数になる}という関係性に基づいています。

\hrule
\vspace{1em}

\section{通常の微分ではなぜ不可能なのか?}

まず、ヘビサイドのステップ関数 $\Theta(x)$ は以下のように定義されます。
\[
\Theta(x) = 
\begin{cases}
1 & (x > 0) \\
0 & (x < 0)
\end{cases}
\]
この関数は $x=0$ の点において不連続であるため、通常の関数の意味では微分不可能です。
$x \neq 0$ の領域では傾きは明らかに $0$ ですが、$x=0$ での「無限大の傾き」とも言える振る舞いを数学的に厳密に記述するために、ディラックのデルタ関数 $\delta(x)$ が導入されます。

\section{超関数としての微分}

ディラックのデルタ関数 $\delta(x)$ は、以下の2つの重要な性質を持つ超関数として定義されます。

\begin{enumerate}
    \item $x \neq 0$ の領域では、常に $\delta(x) = 0$ である。
    \item 全空間で積分すると $1$ になる。
    \[
    \int_{-\infty}^{\infty} \delta(x) \,dx = 1
    \]
\end{enumerate}

この性質を持つデルタ関数を $-\infty$ からある点 $x$ まで積分すると、どうなるかを考えます。

\begin{itemize}
    \item \textbf{$x < 0$ の場合}:\\
    積分範囲にデルタ関数が値を持つ $t=0$ の点が含まれないため、積分値は $0$ です。
    \[
    \int_{-\infty}^{x} \delta(t) \,dt = 0
    \]

    \item \textbf{$x > 0$ の場合}:\\
    積分範囲に $t=0$ の点が含まれるため、デルタ関数の全ての「強さ」を拾うことになり、積分値は $1$ となります。
    \[
    \int_{-\infty}^{x} \delta(t) \,dt = \int_{-\infty}^{\infty} \delta(t) \,dt = 1
    \]
\end{itemize}

この積分結果をまとめると、まさしくステップ関数 $\Theta(x)$ の定義そのものになっていることが分かります。
\[
\int_{-\infty}^{x} \delta(t) \,dt = \Theta(x)
\]

\section{結論:微積分の基本定理}

微分と積分が互いに逆の演算であるという微積分の基本定理の関係は、超関数の世界でも成立します。
上記の積分関係式の両辺を $x$ で微分すると、
\[
\frac{d}{dx} \left( \int_{-\infty}^{x} \delta(t) \,dt \right) = \delta(x)
\]
となります。左辺の被微分関数は $\Theta(x)$ に他ならないため、以下の関係式が導かれます。
\[
\frac{d\Theta(x)}{dx} = \delta(x)
\]

これが、ステップ関数を(超関数の意味で)微分するとデルタ関数になる理由です。

\end{document}
\end{document}